% 导言区 设置文章属性
\documentclass[10pt]{article} % book, report, letter

%在终端用 texdoc 包名 打开对应的文档说明

%设置normal size 为10pt
\usepackage{ctex} % 引入支持中文的包
%提供针对XeTex的改进并加入了XeTex的LOGO
\usepackage{xltxtra}
%图片
\usepackage{graphicx}
%矩阵
\usepackage{amsmath}

%\graphicspath{{figures/}, {otherpath/}} 设置在当前目录下figures中
\graphicspath{{/home/zzy/图片/}, {}}


\newcommand\degree{^\circ} % 定义变量 文中用\degree使用
\newcommand{\myfont}{\textbf{\textsf{Fancy Text}}}


\title{\heiti My First Document} % 支持修改字体
\author{zzy}
\date{\today}

% 正文区
\begin{document}
	\maketitle
	
	ctex具体使用查看文档\\
	$f(x)^\circ$  % $里包含的是数学公式
	$$f(x)$$ %用两个$会另其一行
	\begin{equation} % 生成带编号的数学公式
		AB^2 = BC ^2 + AC ^2
	\end{equation}
	% 字体族设置 (罗马字体、无称线字体、打字机字体)
	% 两种方式使用
	\textrm{Roman Family}
	{\rmfamily test\\}
	\texttt{Typewriter Family}
	{\ttfamily test\\}
	\textsf{Sans Serif Family}
	{\sffamily test\\}
	% 字体系列设置(粗细、宽度)
	\textup{Upright Shape}
	{\upshape test\\}
	\textit{Italic Shape}
	{\itshape test\\}
	\textsl{Slanted Shape}
	{\slshape test\\}
	\textsc{Small Caps Shape}
	{\scshape test\\}
	% 中文字体
	{\songti 宋体\\}
	{\heiti 黑体\\}
	{\fangsong 仿宋\\}
	{\kaishu 楷书\\}
	\textbf{blod\\}
	\textit{中文斜体\\}
	% 字体大小
	{\tiny Hello}
	{\scriptsize Hello}
	{\footnotesize Hello}
	{\small Hello}
	{\normalsize Hello}
	{\large Hello}
	{\Large Hello}
	{\LARGE Hello}
	{\huge Hello}	
	{\Huge Hello}	
	
	% 中文字号设置命令 可以是负数
	\zihao{5} 你好!
	
	% 自定义字体
	\myfont{你好v }
	
	%设定小节
	% 如果用了book可以用\chapter{title} 但此时subsubsection没用的
	\tableofcontents %产生整个的目录
	\section{first}
	添加正文
		\subsection{one}
			\subsubsection{1}
			添加正文,在正文中间插入$\backslash$par \par 换段落
	\section{second}
	\section{third}
	
	\section{空白字符}
	% 空行多个等于一个
	% 自动缩进,不能用空格代替
	% 英文中多个空格处理为1个空格,中文空格会被忽略
	% 汉字与其他字符的间距会自动由XeLaTeX处理
	% 禁止使用中文全角空格
	% 1em
	a\quad b\\
	
	% 2em
	a \qquad b\\
	
	% 约为1/6em
	a \thinspace b \\
	
	%0.5em
	a \enspace b \\
	
	% 空格
	a\ b \\
	
	% 硬空格
	a~b \\
	
	% 1pc=12pt=4.218mm
	a\kern 1pc b\\
	a\kern -1em b\\
	a\hskip 1em b\\
	a\hspace{35pt}b\\
	
	
	% 弹性长度
	a\hfill b	
	% 反斜杠用 \textbackslash
	\textbackslash
	
	
	\section{标识符号}
	\TeX{} \LaTeX{} \LaTeXe{}
	\section{控制符}
	\# \$ \% \{ \} \~{} \_{} \^{} \&
	\section{排版符号}
	\S \P \dag \ddag \copyright \pounds
	\section{插入图片}
	% 导言区: \usepackage{graphix}
	% 语法: \includegraphics[<选项>]{<文件名>}
	% 格式: EPS,PDF, PNG, JPEG, BMP
	
	% \includegraphics[height=2cm, with=2cm or width=0.2\textwidth]{name}
	\section{表格} % 五列的表格分别是左对齐 居中对齐 p指定列 居中对齐 右对齐 竖线来生成竖线
		\begin{tabular}{l | c || p{1.5cm} c r}
			\hline \hline% 生成横线 两个双横线
			姓名 & 语文 & 3 & 4 & 5 \\
			6 & 7 & 8 & 9 & 10
		\end{tabular}
	\section{浮动体}
	\newpage
	% 图片浮动体
	\begin{figure}[htbp] % h 当前位置 t页顶 b页底 p独立一页
		\centering % 环境中居中
		\includegraphics[width=2cm]{loin}
		\caption{ \TeX 系统中的吉祥物--小狮子} \label{flg-lion}%标题
	\end{figure}
	% 表格浮动体
	\begin{table}[htbp]
		\begin{tabular}{l | c || p{1.5cm} c r}
			\hline \hline% 生成横线 两个双横线
			姓名 & 语文 & 3 & 4 & 5 \\
			6 & 7 & 8 & 9 & 10
		\end{tabular}
		\caption{考试水平单}\ref{flg-lion} % 这里会显示1,因为\label设置在图1那
	\end{table}
	\section{数学}
	\subsection{上标}
	$3x^{20}$ % 20要用{}进行分组
	\subsection{下标}
	$2x_2$
	\subsection{希腊字母}
	$\alpha$
	$\beta$
	$\gamma$
	$\epsilon$
	$\pi$
	$\omega$
	\\
	$\Gamma$
	$\Delta$
	$\Theta$
	$\Pi$
	$\Omega$
	\subsection{数学符号}
	$\log$
	$\sin$
	$\cos$
	$\arcsin$
	$\arccos$
	$\ln$
	$\sqrt{x}$
	$\sqrt[3]{x}$
	$3/4$
	$\frac{1}{2}$
	\subsection{行间公式}
	\subsubsection{美元符号}
	交换率是 $$a+b = b+a$$
	\subsubsection{displaymath环境}
	\begin{displaymath}
		a+b=b+a
	\end{displaymath}
	\subsubsection{自动编号公式equation环境}
	\begin{equation}
		a+b = b+a \label{eq:commutative}
	\end{equation}
	\subsubsection{不编号公式equation*环境}
	\section{矩阵}
	$$
	\begin{matrix} %\usepackage{amsmath}
		0 & 1 \\
		1 & 0
	\end{matrix} \qquad
	\begin{pmatrix} %\usepackage{amsmath}
	0 & 1 \\
	1 & 0
	\end{pmatrix} \qquad
	\begin{bmatrix} %\usepackage{amsmath}
	0 & 1 \\
	1 & 0
	\end{bmatrix} \qquad
	\begin{Bmatrix} %\usepackage{amsmath}
	0 & 1 \\
	1 & 0
	\end{Bmatrix} \qquad
	\begin{vmatrix} %\usepackage{amsmath}
	0 & 1 \\
	1 & 0
	\end{vmatrix} \qquad
	\begin{Vmatrix} %\usepackage{amsmath}
	0 & 1 \\
	1 & 0
	\end{Vmatrix} \qquad
	$$
	% \text{test} 可以在数学符号中临时切换成文本
	% 用 \multicolumn{2}{c}{text}合并多列
	\subsection{省略符号} % \dots \ddots \vdots 
	$$
	\begin{bmatrix} % 用 \times 匹配乘法
	0 & \dots & 1 \\
	& \ddots & \vdots  \\
	0 & & 6 \\
	\hdotsfor{4} % 产生跨列省略号
	\end{bmatrix}_{n \times n}
	$$
	产生行内小矩阵\begin{math}
		\left(\begin{smallmatrix} %\usepackage{amsmath} \left(生成左括号
		0 & 1 \\1 & 0
		\end{smallmatrix}\right)
	\end{math}
	% array环境(类似表格)
	\[
	\begin{array}{r|r}
	\frac12 & 0\\
	\hline
	0 & -\frac a{bc} \\
	\end{array}
	\]
	\section{多行公式排版}
	\begin{gather}
		a+b=b+ a \notag\\ % notag不编号
		ab ba
	\end{gather}
	\begin{gather*}
	a+b=b+ a\\
	ab ba
	\end{gather*}
	\begin{equation}
	\begin{split}
	\cos 2x &=\cos^2 x - \sin^2 x\\
	&=2\cos^2 x-1
	\end{split}
	\end{equation}
	
	%case环境
	%类似分段函数
	\begin{equation}
		D(x) = \begin{cases}
		1, & \text{如果} x \in {Q}; \\
		0, & \text{如果} x \in {R} \setminus {Q}.
		\end{cases}
	\end{equation}
	\section{参考文献}
		%\begin{thebibliography}{编号样本}
		%	\bibitem[记号]{引用标志}文献条目1
		%\end{thebibliography}
		\begin{thebibliography}{99}
			\bibitem{article1} 123 \emph 强调内容
		\end{thebibliography}
		\cite{article1} % 引用参考文献
		
\end{document}
