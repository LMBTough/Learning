% 导言区 设置文章属性
\documentclass[10pt]{article} % book, report, letter
%设置normal size 为10pt
\usepackage{ctex} % 引入支持中文的包
%提供针对XeTex的改进并加入了XeTex的LOGO
\usepackage{xltxtra}


\newcommand\degree{^\circ} % 定义变量 文中用\degree使用
\newcommand{\myfont}{\textbf{\textsf{Fancy Text}}}


\title{\heiti My First Document} % 支持修改字体
\author{zzy}
\date{\today}

% 正文区
\begin{document}
	\maketitle
	
	ctex具体使用查看文档\\
	$f(x)^\circ$  % $里包含的是数学公式
	$$f(x)$$ %用两个$会另其一行
	\begin{equation} % 生成带编号的数学公式
		AB^2 = BC ^2 + AC ^2
	\end{equation}
	% 字体族设置 (罗马字体、无称线字体、打字机字体)
	% 两种方式使用
	\textrm{Roman Family}
	{\rmfamily test\\}
	\texttt{Typewriter Family}
	{\ttfamily test\\}
	\textsf{Sans Serif Family}
	{\sffamily test\\}
	% 字体系列设置(粗细、宽度)
	\textup{Upright Shape}
	{\upshape test\\}
	\textit{Italic Shape}
	{\itshape test\\}
	\textsl{Slanted Shape}
	{\slshape test\\}
	\textsc{Small Caps Shape}
	{\scshape test\\}
	% 中文字体
	{\songti 宋体\\}
	{\heiti 黑体\\}
	{\fangsong 仿宋\\}
	{\kaishu 楷书\\}
	\textbf{blod\\}
	\textit{中文斜体\\}
	% 字体大小
	{\tiny Hello}
	{\scriptsize Hello}
	{\footnotesize Hello}
	{\small Hello}
	{\normalsize Hello}
	{\large Hello}
	{\Large Hello}
	{\LARGE Hello}
	{\huge Hello}	
	{\Huge Hello}	
	
	% 中文字号设置命令 可以是负数
	\zihao{5} 你好!
	
	% 自定义字体
	\myfont{你好v }
	
	%设定小节
	% 如果用了book可以用\chapter{title} 但此时subsubsection没用的
	\tableofcontents %产生整个的目录
	\section{first}
	添加正文
		\subsection{one}
			\subsubsection{1}
			添加正文,在正文中间插入$\backslash$par \par 换段落
	\section{second}
	\section{third}
	
	\section{空白字符}
	% 空行多个等于一个
	% 自动缩进,不能用空格代替
	% 英文中多个空格处理为1个空格,中文空格会被忽略
	% 汉字与其他字符的间距会自动由XeLaTeX处理
	% 禁止使用中文全角空格
	% 1em
	a\quad b\\
	
	% 2em
	a \qquad b\\
	
	% 约为1/6em
	a \thinspace b \\
	
	%0.5em
	a \enspace b \\
	
	% 空格
	a\ b \\
	
	% 硬空格
	a~b \\
	
	% 1pc=12pt=4.218mm
	a\kern 1pc b\\
	a\kern -1em b\\
	a\hskip 1em b\\
	a\hspace{35pt}b\\
	
	
	% 弹性长度
	a\hfill b	
	% 反斜杠用 \textbackslash
	\textbackslash
	
	
	\section{标识符号}
	\TeX{} \LaTeX{} \LaTeXe{}
	\section{控制符}
	\# \$ \% \{ \} \~{} \_{} \^{} \&
	\section{排版符号}
	\S \P \dag \ddag \copyright \pounds
	\section{插入图片}
	% 导言区: \usepackage{graphix}
	% 语法: \includegraphics[<选项>]{<文件名>}
	% 格式: EPS,PDF, PNG, JPEG, BMP
	\usepackage{graphicx}
	%\graphicspath{{figures/}, {otherpath/}} 设置在当前目录下figures中
	
\end{document}
